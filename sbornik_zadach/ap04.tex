\chapter{Матеріали до Л/Р \No 4}

\section{Рядки та регулярні вирази}
\subsection{Функції роботи з рядками}
\label{str-func:app}
\nopagebreak[4]


\index{PHP!змінні!рядки!функції}
\begin{longtable}[t]{|l|p{21em}|}

\caption{\space Повний список функцій роботи з рядками} \label{str-func:table}\\
\hline

Функція & Опис \\
\hline \endfirsthead
\caption*{\space Продовження} \\
\hline
Функція & Опис \\
\hline \endhead
\hline \endfoot
addcslashes & екрануючі спецсимволи в стилі мови C \\
addslashes & екрануючі спецсимволи в рядку \\
bin2hex & Перетворює бінарні дані у шістнадцятірічне подання \\
chr & Повертає символ за його кодом \\
chunk\_split & Розбиває рядок на фрагменти \\
convert\_cyr\_string & Перетворює рядок з одного кириличної кодування в інше \\
count\_chars & Повертає інформацію про символи, що входять в рядок \\
crc32 & Обчислює CRC32 для рядка \\
crypt & Необоротне шифрування (хешування) \\
echo & Виводить одну чи більше рядків \\
explode & Розбиває рядок на підрядки \\
fprintf & Записує отформатированную рядок у потік \\
get\_html\_translation\_table & Повертає таблицю перетворень \\
hebrev & Перетворює текст на івриті з логічного кодування у візуальне \\
hebrevc & Перетворює текст на івриті з логічнго кодування у візуальне з перетворенням в переклад \\
htmlentities & Перетворює символи у відповідні HTML теги \\
htmlspecialchars & Перетворює спеціальні символи в HTML теги \\
html\_entity\_decode & Перетворює HTML теги в відповідні символи \\
implode & Об'єднує елементи масиву в рядок \\
localeconv & Повертає інформацію про числові формати \\
ltrim & Видаляє пробіли з початку рядка \\
md5 & Повертає MD5-хеш рядка \\
md5\_file & Повертає MD5-хеш файлу \\
metaphone & Повертає ключ metaphone для рядка \\
nl2br & Вставляє HTML-код розриву рядка перед кожним переведенням рядка \\
number\_format & Форматує число з поділом груп \\
ord & Повертає ASCII-код символу \\
parse\_str & Розбирає рядок у змінні \\
print & Виводить рядок \\
printf & Виводить відформатований рядок \\
quoted\_printable\_decode & розкодує рядок, закодовану методом quoted printable \\
quotemeta & екрануючі спеціальні символи \\
rtrim & Видаляє пробіли з кінця рядка \\
sha1 & Повертає SHA1-хеш рядка \\
sha1\_file & Повертає SHA1-хеш файлу \\
similar\_text & Обчислює ступінь схожості двох рядків \\
soundex & Повертає ключ soundex для рядка \\
sprintf & Повертає відформатований рядок \\
sscanf & Розбирає рядок у відповідності із заданим форматом \\
strcasecmp & Порівняння рядків без урахування регістра, безпечне для даних у двійковій формі \\
strcmp & Порівняння рядків, безпечне для даних у двійковій формі \\
strcoll & Порівняння рядків з урахуванням поточної локалі \\
strcspn & Повертає довжину ділянки на початку рядка, не відповідного  масці \\
stripcslashes & Видаляє екранування символів, вироблене функцією addcslashes () \\
stripos & Повертає позицію першого входження підрядка без урахування регістра \\
stripslashes & Видаляє екранування символів, вироблене функцією addslashes () \\
strip\_tags & Видаляє HTML і PHP теги з рядка \\
stristr & Аналог функції strstr, але незалежний від регістру \\
strlen & Повертає довжину рядка \\
strnatcasecmp & Порівняння рядків без урахування регістра з використанням алгоритму \\
strnatcmp & Порівняння рядків з використанням алгоритму "природнього упорядкування" \\
strncasecmp & порівняння перших n символів рядків без урахування регістра, безпечне для даних у двійковій формі \\
strncmp & порівняння перших n символів рядків без урахування регістра, безпечне для даних у двійковій формі \\
strpos & Знаходить перше входження підрядка в рядок \\
strrchr & Знаходить останнє входження символу в рядок \\
strrev & Перевертає рядок \\
strripos & Повертає позицію останнього входження підрядка без урахування регістра \\
strrpos & Знаходить останнє входження символу в рядок \\
strspn & Повертає довжину ділянки на початку рядка, відповідного масці \\
strstr & Знаходить перше входження підрядка \\
strtok & Розбиває рядок \\
strtolower & Перетворює рядок у нижній регістр \\
strtoupper & Перетворює рядок у верхній регістр \\
strtr & Перетворює задані символи \\
str\_ireplace & Регістро-незалежний варіант функції str\_replace (). \\
str\_pad & Доповнює рядок інший рядком до заданої довжини \\
str\_repeat & Повертає повторювану рядок \\
str\_replace & Замінює рядок пошуку на рядок заміни \\
str\_rot13 & Виконує над рядком перетворення ROT13 \\
str\_shuffle & перемішує символи в рядку \\
str\_split & Розбиває рядок в масив \\
str\_word\_count & Повертає інформацію про слова, що входять в рядок \\
substr & Функція повертає частину рядка \\
substr\_count & Підраховує кількість входжень підрядка в рядок \\
substr\_replace & Замінює частину рядка \\
trim & Видаляє пробіли з початку та кінця рядка \\
ucfirst & Перетворює перший символ рядка в верхній регістр \\
ucwords & Перетворює у верхній регістр перший символ кожного слова в рядку \\
vprintf & Виводить відформатований рядок \\
vsprintf & Повертає відформатований рядок \\
wordwrap & Виконує перенесення рядка на дану кількість символів з використанням символу розриву рядка \\
\hline
\end{longtable}

\subsection{Метасимволи та керуючі конструкцію регулярних виразів у MySQL}
\label{chr-rxp:app}
\index{PHP!регулярні вирази}
У додатку надано перелік метасимволів та керуючих конструкцій для регулярних виразів, що підтримуються у СУБД MySQL.  

\pagebreak[4]
\begin{longtable}[t]{|l|p{20em}|}
\caption{Спецпослідовності регулярних виразів POSIX 1003.2 для MySQL} \label{chr2-rxp:table}\\
\hline

Позначення & Опис \\
\hline

\verb'\t' & символ табуляции. \\
\verb'\f' & конец файла. \\
\verb'\n' & символ перевода строки. \\
\verb'\r' & символ возврата каретки. \\
\verb'\\' & символ обратного слэша \verb'\'. \\


\hline
\end{longtable}


\pagebreak[3]

\begin{longtable}[t]{|l|p{20em}|}

\caption{Опис метасимволів регулярних виразів POSIX 1003.2 для MySQL} \label{chr-rxp:table}\\
\hline

Позначення & Опис \\
\hline




\verb|^| & Відповідає початку рядка. \\

\verb|$| & Відповідає кінцю рядка. \\

\verb|.| & Відповідає будь-якому символу. \\

\verb|a*| & Відповідає будь-якій послідовності з 0 або більше символів <<a>>. \\

\verb|a+| & Відповідає будь-якій послідовності з 1 або більше символів <<a>>. \\

\verb|a?| & Відповідає 0 або 1 символу <<a>>. \\

\verb'de|abc' & Відповідає послідовності <<de>> або <<abc>>. \\

\verb|(abc)*| & Відповідає 0 або більше послідовностям <<abc>>.  \\

\verb|[a-dX],[^a-dX]| & Відповідає будь-якому символу, який є (або не яв ляется, якщо використовується \verb|^|) будь-яким із символів а, Ь, с, d або X. Символ '-' між двома іншими символами утворює інтервал. \\
  
\hline
\end{longtable}


\pagebreak[3]



\begin{longtable}[t]{|l|p{20em}|}

\caption{Класи символів регулярних виразів POSIX 1003.2 для MySQL} \label{chr3-rxp:table}\\
\hline

Позначення & Опис \\
\hline


\verb|[:alnum:]| & алфавітно цифрові символи. \\
\verb|[:alpha:]| & символи алфавіту. \\
\verb|[:blank:]| & символи пробілу і табуляції. \\
\verb|[:cntrl:]| & керуючі символи. \\
\verb|[:digit:]| & десяткові цифри (0-9). \\
\verb|[:graph:]| & графічні (видимі) символи. \\
\verb|[:lower:]| & символи алфавіту в нижньому регістрі. \\
\verb|[:print:]| & графічні або невидимі символи. \\
\verb|[:punct:]| & знаки пунктуації. \\
\verb|[:space:]| & символи пробілу, табуляції, нового рядка або повернення каретки. \\
\verb|[:upper:]| & символи алфавіту в верхньому регістрі. \\
\verb|[:xdigit:]| & шістнадцяткові цифри. \\

\hline
\end{longtable}


\pagebreak[3]



