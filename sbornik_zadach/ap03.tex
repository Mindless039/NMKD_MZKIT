\chapter{Матеріали до Л/Р \No 3}

\section{Змінні оточення web-сервера Apache}
\label{var-apa:app}
\nopagebreak[4]
\index{Web-сервери!Apache!змінні оточення}
\subsection{Формовані сервером змінні}

\textbf{AUTH\_TYPE} Використовується схема аутентифікації. Зазвичай BASIC

\textbf{CONTENT\_LENGTH} Довжина вмісту

\textbf{CONTENT\_TYPE} MIME-тип вмісту

\textbf{GETAWAY\_INTERFACE} Версія CGI, наприклад CGI/1.1

\textbf{PATH\_info} HTTP-шлях до сценарію

\textbf{PATH\_TRANSLATED} Повний шлях до сценарію

\textbf{REMOTE\_ADDR} IP-адреса запитуваного комп'ютера-клієнта

\textbf{REMOTE\_HOST} Доменне ім'я запитувача комп'ютера (якщо доступно). Доменне ім'я визначається веб-сервером за допомогою служби DNS. Директива \verb'HostnameLookups' сервера Apache дозволяє (або забороняє) перетворення IP-адреси в доменне ім'я.

\textbf{REMOTE\_PORT} Порт, закріплений за браузером для отримання відповіді від сервера

\textbf{REMOTE\_USER} Ім'я користувача, що пройшов аутентифікацію

\textbf{QUERY\_STRING} Рядок переданих серверу параметрів

\textbf{SERVER\_ADDR} IP-адресу сервера

\textbf{SERVER\_NAME} Доменне ім'я сервера. Визначається директивою ServerName файлу конфігурації

\textbf{SERVER\_PORT} TCP-порт Web-сервера. Зазвичай 80

\textbf{SERVER\_PROTOCOL} Версія протоколу HTTP. Наприклад, HTTP/1.1

\textbf{SERVER\_SOFTWARE} Програмне забезпечення сервера

\textbf{SCRIPT\_NAME} HTTP-шлях до сценарію

\textbf{SCRIPT\_FILENAME} Файл сценарію в файловій системі сервера (фізичний шлях). Наприклад, \verb'/var/www/cgi-bin/script.cgi'


\subsection{Спеціальні змінні сервера Apache}

\textbf{DOCUMENT\_ROOT} Фізичний шлях до кореневого WWW-каталогу сервера. Наприклад, \verb'/var/www.html/'

\textbf{SERVER\_ADMIN} Адреса електронної пошти адміністратора сервера

\textbf{SERVER\_SIGNATURE} Підпис сервера. Наприклад, <<\verb'Apache/1.3.3 сервера на www.somefirm.com порт 80'>>


\subsection{Змінні HTTP-полів запиту}

\textbf{HTTP\_HOST} Ім'я віртуального хоста, якому адресовано запит

\textbf{HTTP\_USER\_AGENT} Програмне забезпечення віддаленого користувача. Зазвичай ця змінна оточення містить назву і версію браузера

\textbf{HTTP\_ACCEPT} Список підтримуваних клієнтом типів інформації. 

\textbf{HTTP\_ACCEPT\_LANGUAGE} Список підтримуваних мов в порядку переваги, наприклад, RU, EN

\textbf{HTTP\_ACCEPT\_ENCODING} Список підтримуваних методів стиснення

\textbf{HTTP\_ACCEPT\_CHARSET} Список підтримуваних кодувань

\textbf{HTTP\_CONNECTION} Тип з'єднання. Можливі два варіанти:
\begin{list}{•}{•}
\item \verb'Keep-Alive'~--- якщо після відповіді на запит не потрібно розривати з'єднання;
\item \verb'Close'~--- якщо потрібно закрити з'єднання відразу після відповіді на запит.
\end{list}

\textbf{HTTP\_REFERER} Значення поля \verb'REFERER'. У цьому полі браузер передає URL ресурсу, який посилається на наш сервер. Наприклад, якщо користувач перейшов на сайт зі сторінки http://www.somehost.com/page.php, то значення поля \verb'REFERER' буде http://www.somehost.com/page.php.

\textbf{HTTP\_X\_FORWARDED\_FOR} Якщо користувач працює через проксі-сервер, то в цьому полі буде IP-адреса комп'ютера, який звернувся до проксі-сервера. Якщо це поле вже містить значення, то нове значення буде додано через кому.



\subsection{Суперглобальні масиви PHP}
\index{PHP!змінні!суперглобальні масиви}
\label{sup-glob:app}

\textbf{\$GLOBALS}~--- масив всіх глобальних змінних (у тому числі і для користувача).

\textbf{\$\_SERVER}~--- містить безліч інформації про поточний запит і сервер.

\textbf{\$\_ENV}~--- поточні змінні середовища. Їх набір специфічний для кожної конкретної платформи, на якій виконується сценарій.

\textbf{\$\_GET}~--- асоціативний масив з параметрами GET-запиту. У початковому вигляді ці параметри доступні в \verb|$_SERVER ['QUERY\_STRING']| і в \verb|$_SERVER ['REQUEST_URI']| в складі URI.

\textbf{\$\_POST}~--- асоціативний масив значень полів HTML-форми при відправки методом POST.

\textbf{\$\_FILES}~--- асоціативний масив з відомостями про надіслані методом POST файлах. Кожен елемент має індекс ідентичний значенню атрибута \verb'name' у формі і, в свою чергу, також є масивом з наступними елементами:
\begin{enumerate}
\item \verb|$_FILES['name']|~--- вихідне ім'я файлу на комп'ютері користувача.
\item \verb|$_FILES['type']|~--- зазначений агентом користувача MIME~--- тип файлу.
\item \verb|$_FILES['size']|~--- розмір файлу в байтах.
\item \verb|$_FILES['tmp_name']|~--- повний шлях до файлу в тимчасовій папці.
\item \verb|$_FILES['error']|~--- код помилки.
\end{enumerate}

\textbf{\$\_COOKIE}~--- асоціативний масив з переданими агентом користувача значеннями cookie.

\textbf{\$\_REQUEST}~--- загальний масив ввідних даних запиту користувача як в масивах \verb'$_GET, $_POST, $_COOKIE'. Починаючи з версії PHP 4.1 включається і вміст \verb'$_FILES'.


\section{Пріоритети виконання операторів}
\index{PHP!оператори!пріоритети}


\begin{center}
\begin{longtable}[t]{|c|p{25em}|}
\caption{Пріоритети виконання операторів} \label{pr-op:table}\\
\hline

Асоціативність & Оператор \\
\hline \endfirsthead
\caption*{\space Продовження} \\
\hline
Асоціативність & Оператор \\
\hline \endhead
\hline \endfoot
неасоціативна	& \verb|new| \\
права	& \verb|[|\\
неасоціативна	& \verb|++ --| \\
неасоціативна	& \verb|! ~ -(int) (float) (string) (array) (object) @| \\
ліва	& \verb|* / %| \\
ліва	& \verb|+ - .| \\
ліва	& \verb|<< >>| \\
неассоціативна	& \verb|< <= > >=| \\
неассоціативна	& \verb|== != === !==| \\
ліва	& \verb|&| \\
ліва	& \verb|^| \\
ліва	& \verb'|' \\
ліва	& \verb|&&| \\
ліва	& \verb'||' \\
ліва	& \verb|? :| \\
права	& \verb'= += -= *= /= .= %= &= |= ^= <<= >>='  \\
\penalty -10000
ліва	& \verb|and| \\
ліва	& \verb|xor| \\
ліва	& \verb|or| \\
ліва	& \verb|,| \\

\hline
\end{longtable}
\end{center}